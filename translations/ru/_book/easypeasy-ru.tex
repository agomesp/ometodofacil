% Options for packages loaded elsewhere
\PassOptionsToPackage{unicode}{hyperref}
\PassOptionsToPackage{hyphens}{url}
%
\documentclass[
]{book}
\usepackage{lmodern}
\usepackage{amssymb,amsmath}
\usepackage{ifxetex,ifluatex}
\ifnum 0\ifxetex 1\fi\ifluatex 1\fi=0 % if pdftex
  \usepackage[T1]{fontenc}
  \usepackage[utf8]{inputenc}
  \usepackage{textcomp} % provide euro and other symbols
\else % if luatex or xetex
  \usepackage{unicode-math}
  \defaultfontfeatures{Scale=MatchLowercase}
  \defaultfontfeatures[\rmfamily]{Ligatures=TeX,Scale=1}
\fi
% Use upquote if available, for straight quotes in verbatim environments
\IfFileExists{upquote.sty}{\usepackage{upquote}}{}
\IfFileExists{microtype.sty}{% use microtype if available
  \usepackage[]{microtype}
  \UseMicrotypeSet[protrusion]{basicmath} % disable protrusion for tt fonts
}{}
\makeatletter
\@ifundefined{KOMAClassName}{% if non-KOMA class
  \IfFileExists{parskip.sty}{%
    \usepackage{parskip}
  }{% else
    \setlength{\parindent}{0pt}
    \setlength{\parskip}{6pt plus 2pt minus 1pt}}
}{% if KOMA class
  \KOMAoptions{parskip=half}}
\makeatother
\usepackage{xcolor}
\IfFileExists{xurl.sty}{\usepackage{xurl}}{} % add URL line breaks if available
\IfFileExists{bookmark.sty}{\usepackage{bookmark}}{\usepackage{hyperref}}
\hypersetup{
  pdftitle={EasyPeasy},
  pdfauthor={Hackauthor²},
  hidelinks,
  pdfcreator={LaTeX via pandoc}}
\urlstyle{same} % disable monospaced font for URLs
\usepackage{longtable,booktabs}
% Correct order of tables after \paragraph or \subparagraph
\usepackage{etoolbox}
\makeatletter
\patchcmd\longtable{\par}{\if@noskipsec\mbox{}\fi\par}{}{}
\makeatother
% Allow footnotes in longtable head/foot
\IfFileExists{footnotehyper.sty}{\usepackage{footnotehyper}}{\usepackage{footnote}}
\makesavenoteenv{longtable}
\usepackage{graphicx}
\makeatletter
\def\maxwidth{\ifdim\Gin@nat@width>\linewidth\linewidth\else\Gin@nat@width\fi}
\def\maxheight{\ifdim\Gin@nat@height>\textheight\textheight\else\Gin@nat@height\fi}
\makeatother
% Scale images if necessary, so that they will not overflow the page
% margins by default, and it is still possible to overwrite the defaults
% using explicit options in \includegraphics[width, height, ...]{}
\setkeys{Gin}{width=\maxwidth,height=\maxheight,keepaspectratio}
% Set default figure placement to htbp
\makeatletter
\def\fps@figure{htbp}
\makeatother
\setlength{\emergencystretch}{3em} % prevent overfull lines
\providecommand{\tightlist}{%
  \setlength{\itemsep}{0pt}\setlength{\parskip}{0pt}}
\setcounter{secnumdepth}{5}
\usepackage{booktabs}
\ifluatex
  \usepackage{selnolig}  % disable illegal ligatures
\fi
\usepackage[]{natbib}
\bibliographystyle{apalike}

\title{EasyPeasy}
\author{Hackauthor²}
\date{2020-09-29}

\begin{document}
\maketitle

{
\setcounter{tocdepth}{1}
\tableofcontents
}
\hypertarget{preface}{%
\chapter*{Preface}\label{preface}}
\addcontentsline{toc}{chapter}{Preface}

\textbf{Легкий Путь покончить с порнографией}

\textbf{Версия 1.3.6} compiled · Последняя на \href{https://pmohackbook.org}{pmohackbook.org}

Это переписанная версия "взломанной" ПМО книги, адаптация \emph{Легкого способа бросить курить Аллена Карра} для порнографии. Я не оригинальный автор ни одной из этих книг, я Хак-автор второго поколения.

Оригинальная книга, которая расположена на Google Sites, отличный источник для борьбы с порно, и она помогла мне и многим другим. Однако данное издание имеет много плюсов над оригиналом :

\begin{itemize}
\item
  Она с открытым исходным кодом(расположена на git) и лицензией Creative Commons BY-NC-SA 4.0, благодаря чему любой член сообщества может внести свой вклад в развитие книги.
\item
  Переписанна более кратко и точно, однако все равно доносит центральные идеи.
\item
  Множество речевых и грамматических ошибок исправлено.
\item
  Написана с помощью LaTex, благодаря чему достигается элегантное pdf форматирование и легкая поддержка.
\end{itemize}

Читатели обеих версий книги найдут много сходств, однако не смогут не заметить меньшее количество личных анедкотов, смещение к повествованию от третьего лица и гендерную нейтральность. Ссылку на оригинал можно найти на сайте.

Жизнь Аллена Карра невероятно интересная : будучи заядлым курильщиком по сотне сигарет в день, он смог остановиться сразу после открытия \emph{Легкого пути}, и как зацитировано в книге, это "\emph{позволил ему следовать непреодолимому желанию донести свой метод до как можно большего числа курильщиков.}" Его метод для алкоголя и множества других зависимостей стал мировым бестселлером и я настоятельно рекомендую узнать об этом больше.

Его основная работа посвящена рассеиванию страха, вызванного неправильными представлениями и путаницей в отношении биологических процессов и отказа от курения. Поэтому большая часть книги посвещена логическому анализу страхов и фобий, связанными с борьбой с зависимостью, и которые ведут к неудаче на этом пути. Клиника Карра имеют процент успеха свыше 95, с гарантией возврата денег в случае неудачи. Но самое главное, они позволяют поциентам наслаждаться полной жизнью без зависимостей.

Именно потому, что Аллен Карр давно почил и не отнес интернет-порно как зависимость, с которой нужно бороться в том числе в его клиниках, появилась эта книга. Я не монетезирую ее каким-либо способом.

\emph{\textbf{Хак-книга}: Книга основанная на другой книге. Авторские права соблюдены.}

На протяжении этой книги, Я, оригинальный Хак-авторо, и Аллен Карр будем появляться незаметно чтобы дать вам уникальный и убедительный метод чтобы легко и безболезнено бросить.

В цифровую эпоху, порно материалы широко распространены и почти наверняка вы их видели, может быть случайно. По результатам моих неформальных наблюдений (я просил своих друзей прочитать книгу), Легкий Путь одинаково эффективен как для обычного, так и для тяжело зависимого пользователя. Она не ужасно длинная, с высокими шансами обрести свободу, и поэтому я умоляю вас продолжить чтение.

Эта Хак-книга позволит вам:

\begin{itemize}
\item
  Познать что такое онлайн-порно, мастурбация, и биологическое половое влечение и как они работают.
\item
  Видеть порно как наркотик и лечить это соответствующе.
\item
  Рассеивать фантазии во время секса с реальным человеком.
\item
  Уметь мастурбировать без порно или другого человека.
\end{itemize}

Как бы то ни было, для успеха с помощью Легкого Пути вам необходимо

НЕ ПЕРЕСКАКИВАТЬ ГЛАВЫ

Я не могу подчеркнуть, насколько это важно. Это как открыть замок с комбинацией - цыфры должны быть в четкой последовательности, и с зависимостью все точно также.

Я желаю вам удачи, но когда вы поймете главную мысль, она вам больше не понадобиться.

Всего хорошего,\\
Hackauthor²

\hypertarget{introduction}{%
\chapter{Introduction}\label{introduction}}

\textbf{Этот метод излечит порно зависимость.}

Возможно вам кажется что невозможно любому человеку ощутить процесс бросания легким и приятным. Если это так, то Я умолаю вас продолжить чтение - Легкий Путь сработал так же эффективно для других как и для меня.

Легкий путь адоптирован из практики клиник Аллена Карра, где если курильщику не удается бросить, это расценивается как вина клиники в лечении. Схоже, когда пользовтель проваливает попытку бросить порно, ошибочно считать это их собственной виной. Клиники Карра гарантируют возврат денег в случае неудачи, однако неудачных случаев чрезвычайно мало : лишь пять процентов.

Метод, описанный в этой книге:

\begin{itemize}
\item
  Моментальный.
\item
  Одинаково эффективен как для тяжело зависимого, так и для простого пользователя.
\item
  Не вызывает синдрома отмены.
\item
  Не требует силы воли.
\item
  Никакой шок терапии, лекарств, уловок.
\item
  Не заставит заменять одну зависимость другой, такой как переедание, курение или алкоголь.
\item
  Вечный.
\item
  Перманентный.
\end{itemize}

Эта хак-книга даст вам комбинацию для разблокирования замка порно зависимости, но критически важно использовать цифры в нужном порядке. Вам нужно двигаться по течению вместе с книгой, \textbf{не пропуская главы и не прыгая по книге.} Более того, вам даже не нужно сокращать потребление во время чтения.

Классические пути бросить, такие как метод силы воли или замены через порно-диеты(использовать раз в X дней) и уменьшение потребления, которые оба одинаково неэффективны, они не убирают саму причину использования порно. Наоборот, превращение порно в "заветный плод" мало помогает в борьбе с зависимостью.

Множество сайтов углубляются в детали о мозге, поддержанные рецензированными исследованиями об нейромедиаторах и нейропластичности. Пока эти сайты информативные, много людей знают об эректильной дисфункции из-за порно и его высокую способность вызывать привыкание, однако ничего не предпринимают. Пользователи, молодые и зрелые стремятся избегать эту информацию, зная что один взгляд на порно не убьет их. Все больше и больше подростков становятся зависимыми, и с этим нужно что-то делать.

\textbf{Легкий Путь не просто очередной метод, но едиственный осмысленный метод для использования! Конечно же это будет не честно если вы поверите мне сразу; оставьте суждения до конца книги.}

Но в конце концов, лучший показатель это комментарии от реальных пользователей.

\emph{"Я не поверил вашим заявлениям и прощу прощения за то, что сомневался. Это было так же просто как и приятно, как вы и сказали. Я поделился ссылкой на книгу с друзьями, однако я не могу понять, почему они не читают ее."}

\emph{"Друг скинул мнессылку на книгу восемь месяцев назад, Я только что прочитал книгу. Я сожалею только о том, что потерял восемь месяцев."}

Даже неудачливые пользователи оставляют пару строк на подобии :\\
\emph{"У меня еще не получилось, однако этот способ лучше из всех что я знаю."}

Каждый может убедиться, что бросить порно легко, включая вас! Все что вам нужно это прочитать остаток книги без предубеждений. Чем больше вы поймете, чем легче вам будет. Даже если вы не понимаете ни слова, если следовать инструкциям, то бросить окажется просто. Более того, вы не будете жаловаться всю жизнь на то что лишились порно, а к концу книги единственной загадкой будет почему вы занимались этим так долго.

С Легким Путем, есть только две причины провала.

\begin{description}
\item[Провал с выполнением инструкций.]
Некоторые будут раздражены догматичностью, с которой книга говорит о некоторый вещах, таких как уменьшение потребления или использование замены. Я не отрицаю, что много людей бросили с помощью этих уловок, но у них получилось не благодаря, а вопреки. Есть люди которые занимаются любовью стоя на гамаке, но это не самый простой путь. У каждого слова есть свое назначение, сделать освобождение легким и тем самым обеспечить успех. Цифры для открытия замка ловушки зашифрованы в этой книге, но их нужно использовать в правильном порядке, переходя от одной главы к другой и не пропуская их.
\item[Провал с пониманием.]
Не воспринимайте все на слово, размышляя не только о том, что вам говорят, но и о ваших собственных взгядах и взглядах общества на секс, интеренет порно, зависимость. Например, те кто думают что это только привычка, спросите себя полчему другие привычки, некоторые приятные, настолько легко прекратить, в то время как эту ужасную привычку, стоющую вам энергии, времени и мужественности настолько сложно забыть. Те, кто верят что наслаждаются порно, спросите себя, почему некоторые вещи, гораздо более приятные вы можете делать или не делать. Почему вам \emph{приходится} смотреть порно, испытывая панику если его нет?
\end{description}

Легкий Путь собирается дать вам знания о том, насколько это просто и радостно - бросить порно. Как и у многих других, одним из моих самых крупных триумфов в жизни было освобождение из порно ловушки. Никакой необходимости чувствовать себя подавленно, наоборот, вы на пути к тому, о чем мечтают все пользователи планеты: к СВОБОДЕ!

\hypertarget{why-is-it-difficult-to-stop}{%
\chapter{Why is it difficult to stop?}\label{why-is-it-difficult-to-stop}}

All users feel something evil has possessed them. In the early days, it's a simple question of \emph{``I will stop, just not today''}, but eventually we progress to believing we haven't got enough willpower to stop or that there's something inherent in porn we must have in order to enjoy life. Porn addiction can be compared to clawing your way out of a slippery pit: as you near the top, you see the sunshine - but find yourself sliding back down as your mood dips. Eventually you open your browser, and as you masturbate, you feel awful and try to work out why you have to do it.

Ask a user, \emph{``If you could go back to the time before you became hooked, with the knowledge you have now, would you have started using porn?''}

\emph{``NO WAY!''} would be the reply.

Ask the confirmed user, someone who defends internet porn and doesn't believe it causes injury to the brain or downregulation of dopamine receptors: \emph{``Do you encourage your children to use porn?''}

\emph{``NO WAY!''} is again the reply.

Porn is an extraordinary enigma. As said previously, the problem isn't explaining why it's easy to stop, it's explaining why it's difficult. The real problem is explaining why anyone does it \emph{after} getting insights on neurological damage. Part of the reason we start is because of the tens of millions already into it, yet all of those wish they hadn't started in the first place and tell us it's like living life in second gear. We cannot quite believe they are not enjoying it. We associate it with freedom or being 'sex-educated' and work hard to become hooked ourselves. We then spend the rest of our lives telling others not to do it and trying to kick the habit ourselves.

We also spend a significant proportion of our time feeling hopeless and miserable. 'Educating' ourselves with the supernormal makes us prefer and long for these cold images, even when warm, real ones are available! Through the constant surge and fall of dopamine induced by PMO, we sentence ourselves to a lifetime of irritability, anger, stress, fatigue, and sexual dysfunction. Using porn, with its absence of the best parts of sex and connection, we end up feeling miserable and guilty.

In fact, reading about internet pornography's addictive and destructive capabilities here and on other sites makes us more nervous and hopeless! What sort of hobby is it that when you're doing it, you wish you weren't, and when you aren't, you crave it? Users despise themselves every time they read about hypofrontality and desensitisation, every time they use behind their trusting partner's back, every time they can't bring themselves to exercise after a daytime session. An otherwise intelligent and rational human being spends all their days in contempt. But worst of all, what do users get from having to endure life with these awful black shadows at the back of their mind? \textbf{Absolutely nothing!}

You might be thinking \emph{``That's all very well, I know this, but once you're hooked on these things it's very difficult to stop.''} But why is it so difficult? Some say it's because of the powerful withdrawal symptoms, but as you'll come to learn, the actual withdrawal symptoms are so mild that you should be aware of PMOers who have lived and died without realising they're drug addicts.

Some say internet porn is free and hence humankind should claim this biological bonanza, but this is untrue---it's addictive and acts just like any drug. Ask a user that swears they only enjoy 'erotica' like Playboy magazines if they've ever crossed the line to 'unsafe porn' and if completely honest, they'd the times they'd unwittingly rationalized doing so, rather than not use anything at all.

Enjoyment has nothing to do with it either: I enjoy crayfish, but I never got to the point where I had to have crayfish every day. With other things in life, we enjoy them while we're doing them, but we don't sit around feeling deprived when we're not.

Some say:\\
\emph{``It's educational!''} So when is your graduation?\\
\emph{``It's sexual satisfaction!''} So why do it alone instead of finding a partner and saving it for them?\\
\emph{``It's a feeling of release!''} Release from the stresses of real life? Porn won't remove the source of the stress, but it does add to it.

Many believe that porn relieves boredom, which is also a fallacy. Boredom is a frame of mind. Porn will habituate you to novelty-seeking in no time, causing you to become increasingly bored until you finally participate in that wild-goose chase for just the right clip, causing you to become increasingly wired to seek anything that evokes novelty, strong emotion, and eventually, outrageous shock value.

Some say they only do it because their friends and everyone they know do it. If so, pray that your friends don't start cutting their heads off to cure a headache! Most users who think about it come to conclude that it's just a habit. This is not really an explanation, but having discounted all the usual, rational explanations, it appears to be the only remaining excuse. Unfortunately, it's equally illogical. Every day of our lives we change habits, some of them very enjoyable. We've been brainwashed to believe that PMO is a habit and that habits are difficult to break.

Are habits difficult to break? Drivers in the US are in the habit of driving on the right hand side of the road, yet when travelling overseas they break the habit with hardly any aggravation whatsoever. It is clearly a fallacy that habits are hard to break. We make and break habits every day of our lives. So why do we find it difficult to break a habit that makes us feel deprived when we don't have it, guilty when we do, one that we would love to break anyway, when all we have to do is \emph{stop doing it?}

The answer is that porn isn't habit, \textbf{it's addiction!} That's why it appears to be so difficult to 'give up'. Most users don't understand addiction and believe that they get some genuine pleasure or crutch from porn. They believe they're making a genuine sacrifice if they quit.

The beautiful truth is that once you understand the true nature of porn addiction and the reasons why you use it, you'll stop doing it, just like that. Within three weeks, the only mystery will be why you found it necessary to use porn as long as you have and why you can't persuade other users \emph{how nice it is to not be a PMOer!}

\hypertarget{the-easy-method}{%
\chapter{The Easy Method}\label{the-easy-method}}

This book's objective is directing you into a new frame of mind. In contrast to the usual method of stopping - whereby you start off the feeling of climbing Mount Everest and spend the next few weeks craving and feeling deprived - you start right away with a feeling of elation, as if cured of a terrible disease. From then on, the further you go through life, the more you will look at this period of time and wonder how you ever used any porn in the first place. You will look at other porn users with pity, as opposed to envy.

Provided that you're not someone who had never become addicted (reading for your significant other) or had quit (or is in the fasting days of a ``porn diet''), it's essential to keep using until you have finished the book completely. This may appear to be a contradiction, and this instruction to continue masturbating to porn causes more objection than any other, but as you read further your desire to use porn will gradually be reduced. \textbf{Take this instruction seriously: Attempting to quit early will not benefit you.}

Many don't finish the book because they feel they have to give something up, some even deliberately only reading one line per day in order to postpone the evil event. Look at it this way, what have you got to lose? If you don't stop at the end of the book, you're no worse off than you are now. It's by definition a Pascal's Wager, a bet taken where you have nothing to lose and high chances of large gains.

Incidentally, if you haven't watched porn for a few days or weeks, but aren't sure whether you're a porn user, ex-user, or a non-user, then don't use porn to masturbate whilst reading. In fact, you're already a non-user, but we have to let your brain catch up with your body. By the end of the book, you'll be a happy non-user. EasyPeasy is the complete opposite of the normal method, where one lists the considerable disadvantages of porn and says:\\
\emph{``If only I can go long enough without porn, eventually the desire will go and I can enjoy life again, free of slavery.''}\\
This is the logical way to go about it, with thousands stopping every day using this method. However, it's very difficult to succeed for the following reasons:

\textbf{Stopping PMO isn't the real problem.}
Every time you finish your session, you've stopped using it. You may have powerful reasons on the first day of your once-in-four porn diet to say \emph{``I don't want to use porn, or even masturbate any more.''} All users do, and their reasons are more powerful than you can possibly imagine. The real problem is day two, ten, or ten-thousand where in a weak moment you'll have `just one peek', want another, and suddenly you're an addict again.

\textbf{Awareness of the health risks generates more fear, making it more difficult to stop.}
Tell a user it's destroying their virility and the first thing they'll do is reach for something to surge their dopamine: a cigarette, alcohol, or even firing up the browser to search for porn.

\textbf{All reasons for stopping actually make it harder.}
This is due to two reasons. First, we're continually being forced to give up our 'little friend' or some prop, vice, or pleasure (whichever way the user perceives it). Second, they create a ``blind''. We do not masturbate for the reasons we should stop. The real question is, why do we want or need to do it?

With EasyPeasy, we (initially) forget the reasons we'd like to stop, face the porn problem and ask ourselves the following questions:

\begin{enumerate}
\def\labelenumi{\arabic{enumi}.}
\item
  What is porn doing for me?
\item
  Am I actually enjoying it?
\item
  Do I really need to go through life sabotaging my mind and body?
\end{enumerate}

The beautiful truth is that \emph{all porn} does absolutely nothing for you whatsoever. Let me make it quite clear, it's not that the disadvantages of being a user outweigh the advantages, it's that there are \textbf{\emph{zero}} advantages to looking at internet porn.

Most users find it necessary to rationalise why they use porn, but the reasons they come up with are all fallacies and illusions.

First, we'll remove these fallacies and illusions. In fact, you'll soon realise there is nothing to give up. Not only that, but there are marvelous, positive gains from being a non-PMOer, with well-being and happiness only two of these gains. Once illusions that life will never be quite as enjoyable without porn is removed - realising that not only is life just as enjoyable without it but infinitely more so - and once feelings of being deprived or missing out are eradicated, we'll go back to reconsider increased well-being and happiness---and the dozens of other reasons for quitting porn. These realisations will become positive additional aids to help you achieve what you really desire: enjoying your life free from the slavery of porn addiction!

\hypertarget{the-sinister-trap}{%
\section{The Sinister Trap}\label{the-sinister-trap}}

Internet porn is the most subtle, sinister trap that man and nature have combined to devise; it's the only trap in nature whose setup doesn't require hard work. Some of us are even warned about the dangers, but we can't believe how they aren't enjoying it. But what gets us into it in the first place? Typically, free samples from amateurs and professionals who share. That's how the trap is sprung, your first 'peek' has stains and holes with most thumbnails on any porn page being amateurish and home-made clips of unknown models. If the first timer's gaze was filled only with angelic beauties and professional models then alarm bells would ring.

Due to this mismatch in clips, our young minds are reassured we'll never become hooked, thinking because we don't enjoy them, we can stop whenever we want to. As intelligent human beings, we'd then understand why half the adult population was systematically addicted to something cutting down their very potential to perform what they're viewing. Curiosity brings us closer to their doorsteps, but not daring to click on some thumbnails, fearing they'd make you ill.~If you accidentally clicked on one, your only desire being getting away from the page as soon as possible.

We then spend the rest of our lives trying to understand why we do it, telling children not to start, and at odd times trying to escape ourselves. The trap is designed such that we try and stop only due to an 'incident', whether sexual performance, loss of a career or relationship, shortage of drive or just plain feeling like a leper. As soon as we stop, we have more stress due to withdrawal pangs with the method we relied on to remove that stress unavailable.

After a few days of torture we come to the decision that we've picked the wrong time to quit, deciding we'll wait for periods without stress, which upon arriving removes our reason for initially stopping. Of course, that period will never arrive as we internally believe our lives tend to become more and more stressful. Leaving the protection of our parents, stresses such as jobs, homemaking, mortgages, babies, bigger houses and more babies crowd our lives. This is an illusion, the truth being that the most stressful parts of any creature's life are early childhood and adolescence.

We tend to confuse responsibility and stress. A user's life - like a drug addicts - automatically becomes more stressful because porn doesn't relax you or relieve stress, as some try to make you believe. It's just the reverse, causing you to become more stressed as you continue using, piling more straw onto the camel's back. Even users who kick the habit (most do one or more times throughout their lives) can lead perfectly happy lives yet suddenly become hooked again. Wandering into the pornographic maze, our minds become hazy and we spend the rest of our lives trying to escape. Many do succeed, only to fall into the sinister trap at a later date.

Porn addiction is a complex and fascinating puzzle, and much like a Rubik's Cube, practically impossible to solve. But if you have the solution, it's simple and fun! EasyPeasy contains the solution to this puzzle, leading you out of the maze, never wandering in again. All you have to do is follow the instructions. However, if you take a wrong turn, the rest of the instructions are pointless.

Anyone can find it easy to stop, we must first establish facts. No, not facts designed to scare you, there's already more than enough information out there. If that was going to stop you, you'd have already stopped. But why do we find it difficult to stop? Answering this requires us to know the real reason we're still using porn, boiling down to two factors. They are:

\begin{itemize}
\item
  Nature and internet porn.
\item
  Brainwashing.
\end{itemize}

Porn users are intelligent, rational human beings. They know they're taking enormous future risks so they spend lots of time rationalising their 'habit'. But porn users in their hearts know they're fools, knowing they had no need to use porn before becoming hooked. Most remember that their first 'peek' was a mix of revulsion and novel curiosity. They then specialise in locating, filtering and bookmarking sites, working hard to become hooked.

Most annoyingly, there's the sense that non-addicts - most women, older guys and people living in countries where high-speed internet porn is unavailable - aren't missing out on anything and find the situation laughable. By dismantling these factors in the next chapters, you too will understand the sinister trap!

  \bibliography{book.bib,packages.bib}

\end{document}
