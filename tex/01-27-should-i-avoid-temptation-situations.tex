\documentclass[easypeasy.tex]{subfiles}
\chapter{Should I avoid temptation situations?}
\begin{document}

The advice has been direct so far and ask you to treat it as instruction rather than suggestion. There are sound, practical reasons for this advice and those reasons have been backed up by thousands of case studies. On the question of whether or not to try and avoid temptation, this isn't the case. Each PMOer will need to decide for themselves. However, two helpful suggestions can be made to assist you through this process. It is fear that keeps us PMOing for all our lives and this fear consists of two distinct phases.

\textbf{Phase One - How can I survive without PMO?}

This fear is the panicky feeling the PMOer gets when they're single or have an asexual, uninterested or unavailable partner. The fear isn't casused by withdrawal pangs but is the psychological fear of dependency, being unable to survive without sex and orgasm. It peaks when you're on the verge of quitting, when your withdrawal pangs are at their lowest. It's the fear of the unknown, the sort of fear that people have when they're learning to dive.

The diving board is one foot high but seems to be six feet high. The water is six feet deep but appears to be one foot deep. It takes courage to launch yourself, convinced that you are going to smash your head. The launching is the hardest part, if you find the courage to do it, the rest is easy! This explains why many strong-willed PMOers have never attempted to stop or can survive for only a few days when they do. In fact, there are some PMOers on a porn diet who, when they decide to stop, actually binge and escalate to harsher clips more quickly than if they hadn't decided to stop. This decision causes panic, which is stressful and triggers a cue to take a trip to the harem. But now you can't take one, leading to thoughts of deprivation and compouding stress.

The trigger activates quickly when the fuse blows and you fire up the browser. Don't worry, the panic is just psychological. It's the fear you're dependent. The beautiful truth is that you aren't, even when you're still addicte. Don't panic and launch yourself.

\textbf{Phase Two - Longer Term Fear}

The second phase is long term, involving the fear that certain situations in the future will not be enjoyable or you won't be able to cope with trauma without PMO. Don't worry, if you can launch yourself you will find the opposite to be the case. The avoidance of temptation itself falls into two catagories.
  \begin{enumerate}
  \item I'll subscribe to a porn diet of once in four days. I'll feel more confident knowing that I can go online if it gets difficult. If I fail it's okay, I'll just add additional days to my next cycle.

The failure rate with people who do this is far higher than those two quit altogether. This is mainly due to the fact that if you're having a bad moment during the withdrawal period, it's easy to hop on the browser and visit the harem with the above excuses. If you have the indignity of clearly breaking your own rules like a shattered glass window, you are more likely to overcome the temptation. In any event, the pang would probably already had passed if you had delayed it. However, the main reason for the high failure rate in these cases is that the PMOer did not feel completely committed to stopping in the first place. Remember the two essentials to succeed are:
\begin{itemize}
  \item Certainty
  \item "Isn't it marvellous that I don't need to PMO anymore?"
\end{itemize}

In either case, why on earth do you need to PMO? If you still need to visit your harem, re-read the book first. It means something hasn't quite gelled. Take the time to kill the big brainwashing monster in your mind stone dead.

  \item "Should I avoid stressful or social occasions during the withdrawl period?" 

In the case of stressful situations, yes. There's no sense putting undue pressure on yourself. In the case of social events, like bars or clubs, the advice is the reverse. Go out and enjoy yourself straight away, you don't need the sex or the propagative side of sex even while you're addicted to porn. Go out and rejoice in the fact that you don't have to have sex or propagative sex, it will quickly prove to you the beautiful truth that life is so much better without these pressures. Just think of how much better it will be when the little monster has left you, together with those needy thoughts.
\end{enumerate}
\end{document}
