\documentclass[easypeasy.tex]{subfiles}
\chapter{Nature}
\begin{document}

Internet porn works by hijacking natural reward mechanisms designed to keep you reproducing for as long as possible, the instant and highly accessible novelty of internet porn keeping the brain's reward mechanism producing dopamine for much longer than would normally be possible. Dopamine is a neurotransmitter that encourages action, the real pleasure produced by opioids. More dopamine, more opioids, more action. Without dopamine, actions such as eating don't feel pleasurable and aren't completed, with foods high in fat and sugar producing the highest chemical release.

Porn triggers a flood of dopamine, so the first time you see porn, you act, orgasming and triggering another flood of opiates. The brain, incentivised to get as much dopamine as possible, stores this as a script for easy recall, strengthening the neural pathways by releasing DeltaFosB. In future, a cue such as a sexy commercial, alone time, stress or feeling a little down, the brain calls up the pathway and you're ready to take a ride on the 'water slide'. Every time you repeat this, more DeltaFosB is released and the water slide is greased, alive and easier to slide the next time.

The brain has a self correcting system whereby the number of dopamine and opioid receptors are limited when frequent and daily flooding of dopamine is detected. Unfortunately, these receptors are also needed to keep us motivated to handle stresses in day to day life. The nominal amounts of dopamine produced by natural causes simply don't compare and aren't as efficiently absorbed with decreased receptors, leading you to feel more stressed and irritated than normal. This process is known as desensitisation.

In this cycle, you crossed the 'red line' and triggered emotions such as guilt, disgust, embarrassment, anxiety and fear, which in turn raise dopamine levels even higher causing the brain to misinterpret these feelings as sexual arousal.

As time passes, not only is the brain desensitised to previous clips it has seen, but also to similar genres and shock level. This lower motivation will trigger a feeling of lower satisfaction as our brains engage in constant rating, pushing you to find a clip to sate the hunger. So you seek more novelty, clicking on the amateurish, shock inducing clip on the front page of the site that you confidently said you wouldn't on your first visit.

\textit{"For in the dew of little things the heart finds it's morning and is refreshed" - Kahlil Gibran}

A fleeting feeling of security is all that is needed to get through a rough spot in life, but will your desensitised brain be able to catch that drop of de-stressor that a non-PMOers brain is able to use?

Dopamine flooding, like quick acting drug, falls quickly to induce withdrawal pangs. Many PMOers have the illusion that these pangs are the terrible trauma that they suffer when they try or are forced to stop. They're in fact, primarily mental. The user is feeling deprived of their pleasure or prop.

\section{The Little Monster}
The actual pangs of withdrawal from PMO are so subtle that most users have lived and died without realising they're drug addicts. Many PMOers have a fear of drugs, yet that's exactly what they are, drug addicts. Fortunately, it's an easy drug to kick, but you first need to accept that you are in fact, addicted. There's no physical pain in the withdrawal from PMO, merely an empty, restless feeling of something missing, which is why many believe it's something to do with their hands. Prolonged, the feeling becomes nervousness, insecurity, agitation, low confidence and irritability. It's like hunger, for a poison.

Within seconds of engaging into PMO, dopamine is supplied and the craving ends, resulting in a feeling of fulfillment as you whiz down the water slide. In early days, the withdrawal pangs and their subsequent relief are so slight that we were unaware of them. When we become regular users, we believe it's because we've come to enjoy them or gotten into the 'habit'. The truth is that we're already hooked, but we don't realise it. The little monster is already in our brains and every once and a while we take a trip down the water slide to feed it.

All PMOers begin seeking porn for irrational reasons. The \textit{only} reason anybody continues PMOing, whether they're a casual or heavy user, is to feed that little monster. The whole conundrum is a series of cruel and confusing punishments, but perhaps the most pathetic aspect is the sense of enjoyment a PMOer gets from a session, trying to get back to the sense of peace, tranquility and confidence their body had before they became hooked in the first place.

\section{The Annoying Alarm}
You know that feeling when a neighbour's home alarm has been ringing all day, or some other minor persistent aggravation? Then, the noise suddenly stops and a marvelous feeling of peace and tranquility washes over you. This isn't really peace, but an ending of aggravation. Before we start the next PMO session our bodies are complete, but we then begin forcing our brains to pump dopamine into the body and when we're done and the dopamine begins to leave, suffering withdrawal pangs. These aren't physical pain, just an empty feeling. We aren't even aware it exists, but it's like a dripping tap inside our bodies.

Our rational minds don't understand it, but they don't need to. All we know is that we want porn and when we masturbate the craving goes. However, the satisfaction is fleeting because in order to relieve the craving you have to get more porn. As soon as your orgasm, the craving starts again and the trap continues to hold you. A feedback loop, unless you break it!

The PMO trap is similar to wearing tight shoes just to obtain the pleasure of taking them off. There are three primary reasons why PMOers can't see it this way.

\begin{enumerate}
  \item From birth, we've been subjected to massive amounts of brainwashing telling us that internet porn is simply another modern development that replaced the print version of porn. This fallacy is packaged with the truth that masturbation isn't harmful, so why shouldn't we believe them?

  \item Because physical dopamine withdrawal involves no actual pain, merely an empty insecure feeling inseparable from hunger and normal stress, the feeling manifests into a PMO session as those are the very times we tend to seek internet porn. We tend to regard this feeling as normal.

  \item However, the main reason that PMOers fail to see internet porn in it's true light is because it works back to front. It's when you're \textit{not} consuming it that you suffer the empty feeling. Because the process of getting hooked is incredibly subtle and gradual in the early days, we regard the empty feeling as normal and don't blame it on the previous PMO session. The moment you fire up the browser and begin your session, you get an immediate boost, becoming less nervous or more relaxed and internet porn gets the credit.
  \end{enumerate}

This 'back to front' reverse process makes all drugs difficult to kick. Picture the state of panic of a heroin addict without any heroin, now picture the utter joy of when they can finally plunge a needle into their vein. Non-heroin addicts don't suffer that panicked feeling.

The heroin doesn't relieve the feeling, it causes it. Similarly, non-PMOers don't suffer the empty feeling of needing internet porn or starting to panic when they're offline. Non-PMOers can't understand how PMOers could possibly obtain pleasure from two dimensional videos with muted sounds and abnormal body proportions. Eventually, PMOers can't understand either.

We talk about internet porn being relaxing or satisfying. But how can you be satisfied unless you were dissatisfied in the first place? A non-PMOer don't suffer from this unsatisfied state, completely relaxed after a no-sex date, while the PMOer isn't until they have satisfied their 'little monster'.

\section{A pleasure or a crutch?}
An important reminder, the main reason that PMOers find it difficult to quit is due to the belief they're giving up a genuine pleasure or crutch. It's essential to understand that you're giving up \textit{absolutely nothing} whatsoever. The best way to understand the subtleties of the PMO trap is to compare it with eating. The habit of regular meals causes us to not feel hungry between, only being aware of hunger if the meal is delayed. There's no physical pain, just an empty insecure feeling we recognise as hunger. The process of satisfying our hunger is a very pleasant experience.

PMOing appears to be almost identical, but it's not. Like hunger, there's no physical pain and the reward mechanism behaves in a similar way, but it's this similarity to eating that tricks the PMOer into believing there's a genuine pleasure or crutch. Although eating and PMOing appear to be very similar, in fact they're exact opposites.

\begin{itemize}
  \item You eat to survive and to energise your life, whereas PMOing dims and cuts down your mojo.
  \item Food genuinely tastes good and eating is a genuinely pleasant experience that we can enjoy throughout our lives. PMOing involves self-sabotaging the happiness receptors and thus destroying your chances to cope and feel happy.
  \item Eating doesn't create hunger and genuinely relieves it, whereas the first PMO starts the craving for dopamine and each subsequent PMO. Far from relieving it, ensuring suffering for the rest of your life.
  \end{itemize}

Is eating a habit? If you think so, try breaking it completely! To describe eating as a habit would be like describing breathing as a habit, both are essential for survival. It's true that people are have the habit of satisfying their hunger at different times with varying types of food, but eating itself isn't a habit. Neither is PMO. The only reason a PMOer fires up the browser is to try and end the empty feeling the previous session created, at different times with varying escalating genres.

On the internet, PMO is frequently referred to as a habit and for convenience EASYPEASY also refers to the 'habit'. However, be constantly aware that PMO isn't a habit, it's \textbf{drug addiction!} When we start to PMO, we have to force ourselves to cope with it. Before we know it, we're escalating into more bizarre and shocking porn. The thrill is in the hunting, not the killing, dopamine rapidly leaving the body after orgasm, explaining why PMOers want to 'edge' (delaying orgasm), flicking between multiple browser windows and tabs.

\section{Crossing the red line}
As with any other drug, the body tends to develop immunity to the effects of the same old clips, our brain wanting more or something else. After a short period of watching the same clip it ceases to completely relieve the withdrawal pangs that the previous session created. There's a tug of war occurring in this porn paradise, you want to stay on the safe side of your 'red line', but your brain is asking you to click on the forbidden fruit clip.

You feel better after engaging in this PMO session, but you're more nervous and less relaxed than someone who never started, even though you're living in a supposed porn paradise. This position is even more ridiculous than wearing tight shoes because as you go through life an ever increasing amount of discomfort remains after taking the shoes off. Because they know the little monster has to be fed, the PMOer themselves will decide when, tending to be on four types of occasions or a combination of them.\\
  Boredom / Concentration - Two complete opposites!\\
  Stress / Relaxation - Two complete opposites!

What magic drug can suddenly reverse the very effect it had minutes before? The truth is, PMO neither relieves boredom and stress or promotes concentration and relaxation. If you think about it, what other types of occasions are there in our lives, bar sleep? If you have ideas of toning down to other types of 'realistic' or 'soft' genres of porn, the content of this book applies to all porn, print, webcams, pay-per-views, chat, live shows, ect. The human body is the most sophisticated object on the planet, but no species, even the lowest amoeba or worm, survives without knowing the difference between food and poison.

Through natural selection our minds and bodies have developed techniques for rewarding actions that multiply and sustain humanity. They're not prepared for supernormal stimulai that are bigger, brighter and edgier than anything found in nature, even the most muted two dimensional image will cause us to become aroused. But keep looking at the same image repeatably and you won't be. In real life, checks and balances ensure that you do something else but internet porn has no such thing, causing you to spend your life in a virtual harem!

It's a fallacy that physically and mentally weak people become PMOers, the lucky ones are those who find their first instance repulsive and are cured for life. Alternatively, they aren't mentally prepared to go through the severe learning process of fighting to get themselves hooked, fear of 'getting caught' or not technical enough to operate browser privacy settings. Perhaps the most tragic part of the whole business relates to teenagers, skilled in finding material and covering tracks, start in increasing number.

Enjoying internet porn is an illusion. Jumping from genre to genre, merely keeping our novelty 'monkey' within the 'red line' of 'safe' porn genres to get our dopamine fix. Like heroin addicts, all they're  really enjoying is the ritual of relieving those pangs.

\section{The High From the Dance Around The Red Line}
Even the one clip on lingered on, the PMOer constantly teaches themselves to filter out the bad and ugly portions of a porn clip. Even if it's solo, still filtering on the body parts that appeal to you the most. In fact, some take pleasure in this dance around the red line, finding excuses to declare they like the 'soft stuff', unaddicted to supernormal stimulai. But ask a user who believes they stick to a certain actor or genre, \textit{"If you cannot get your normal brand of porn and can only obtain an unsafe genre, do you stop masturbating?"}

No way! A PMOer will masturbate to anything, escalating genres, differences in sex-orientation, look-alike performers, dangerous settings, shocking relationships, anything to sate the little monster. To begin with they taste awful, but given enough time you'll learn to like them. PMOers will seek empty-fulfillment after having real sex, after a long work day, fever, colds, flu, sore throats and even admitted in hospitals.

Enjoyment has nothing to do with it, if sex is wanted, then it makes no sense to be with your laptop. Some PMOers find it alarming to realise they're drug addicts and believe it'll make it even more difficult to stop. In fact, this is good news for two important reasons.
    \begin{enumerate}
      \item The reason why most of continue PMOing is because although we know the disadvantages far outweigh the advantages, we believe there's something in the porn that we actually enjoy or it's some sort of prop. We're under the illusion that after we stop PMOing there will be a void, certain situations in our lives never being quite the same. The fact is, PMO not only gives nothing, it only subtracts.

      \item Although internet porn is the most powerful trigger for novelty and sex based dopamine flooding, because of the speed you become hooked, you're never badly hooked. The actual withdrawal pangs are so mild that most PMOers have lived and died without realising they've suffered them.
    \end{enumerate}

Why is it then that many PMOers find it so difficult to stop, going through months of torture, spending the rest of their lives pining for it at odd times? The answer is the second reason, the brainwashing. The neurotransmitter addiction is easy to cope with, most PMOers going for days without online porn on business trips or travel, unaffected by withdrawal pangs. Their little monster knows that you'll open your laptop as soon as you return to your hotel room. You can survive your obnoxious client and your megalomaniac manager, knowing the fix is there for your taking.

\section{The Smokers Analogy}
A good analogy is of the cigarette smoker, if they went ten hours of the day without a cigarette they'd be tearing their hair out. Many smokers will buy a new car and refrain from smoking in it. Many will visit theatres, supermarkets, churches and being unable to smoke causes them no problems. Even on trains and airplanes there've been no riots. Smokers are almost pleased for something or someone to stop them smoking.

PMOers will automatically refrain from using internet porn in the home of their parents during family gatherings and other events with little discomfort. In fact, most PMOers have extended periods during which they abstain without effort. The neurological little monster is easy to cope with, even when you're still addicted. There are millions of PMOers who remain casual users all their lives, they're just as addicted as the heavy PMOer. There are even heavy PMOers who've kicked the addiction but have an occasional peek, greasing the water slide to ride down it at the next dip in mood.

As said previously, the actual porn addiction isn't the main problem, simply acting as a catalyst to keep our minds confused over the real problem - brainwashing. Don't think the bad effects of internet porn are exaggerated however, if anything, they're sadly understated. Occasionally, rumors circulate that the neural pathways created are there for life, with the right mix of chance and stimulai sending you down the life ruining water slide again, but these are untrue. Our brains and bodies are miraculous machines, recovering within a matter of weeks.

It's never too late to stop, a quick browse of online communities will show you people of all ages rebooting their (and their partners) lives. As with anything humans do, some take it to the next level, practising semen retention and Karezza, differentiating between the amative and propagative sides of sex and making their partners happier than ever before.

It may be of consolation to lifelong and heavy PMOers that it's just as easy for them to stop as casual PMOers, in a peculiar way, it's easier. The further it drags you down, the greater the relief. When I stopped I went straight to \textit{zero} and didn't have one bad pang. In fact, it was actually enjoyable, even during the withdrawal period.

But first, we must remove the brainwashing.
\end{document}
